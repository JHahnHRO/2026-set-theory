\begin{definition}
    Let $A$ be a well-ordered set. An \emph{initial segment} is a subset $S\subseteq A$ such that
    \[\forall \alpha\in S, \alpha'\in A: \alpha'\leq \alpha \implies \alpha'\in S\]
\end{definition}

\begin{remark}
    Let $A$ be any well-ordered set.

    $\emptyset$ and $A$ itself are initial segments. Sets of the form $A_{<\alpha}$ and $A_{\leq\alpha}$ are initial segments.

    Any \emph{proper}, initial segment $S\subsetneq A$ is of the form $A_{<\alpha}$ for a unique $\alpha\in A$, namely the smallest element of $A\setminus S$.

    Obviously, $ \alpha'\leq\alpha \iff A_{<\alpha'} \subseteq A_{<\alpha}$.
\end{remark}

\begin{lemma}
    \begin{enumerate}
        \item For any two well-ordered sets $A,B$ there is at most one isomorphism $A\to B$.
        \item The only initial segment of $A$ isomorphic to $A$ is $A$ itself. In particular: All initial segments of $A$ are pairwise non-isomorphic.
    \end{enumerate}
\end{lemma}
\begin{proof}
    a. Transfinite induction along $A$: Let $f,g:A\to B$ be two isomorphism, $\alpha\in A$ and assume $f(\beta)=g(\beta)$ for all $\beta < \alpha$. Set $A':=A_{<\alpha}$.

    Then by construction $\alpha$ is the smallest element of $A\setminus A'$ and therefore the isomorphisms $f$ and $g$ must map it to the smallest element of $B\setminus f(A')$ and $B\setminus g(A')$ respectively. But by assumption, $f(A')=g(A')$ so that $f(\alpha)=g(\alpha)$.

    Therefore, $f(\alpha)=g(\alpha)$ for all $\alpha\in A$.

    \medskip
    b. Assume $f: A\to A_0$ is an isomorphism onto an initial segment $A_0\subsetneq A$. If there was an $\alpha\in A\setminus A_0$, then $f(\alpha) < \alpha$. But then by induction $f^{n+1}(\alpha)<f^n(\alpha)$ for all $n\in\IN$ so that we get an infinite descending sequence in $A$ which does not exist because $A$ is well-ordered. Therefore $A_0=A$.
\end{proof}

\begin{remark}
    This means that we can strengthen the previous observation: $\alpha'\leq\alpha$ iff $A_{<\alpha'}$ is merely isomorphic to an initial segment of $A_{<\alpha}$.
\end{remark}

\begin{theorem}[\enquote{The class of well ordered sets is well-ordered}]
    Given two well-ordered sets $A,B$ exactly one of the following happens:
    \begin{enumerate}
        \item $A$ is isomorphic to a proper initial segment of $B$
        \item $A\isomorphic B$
        \item $B$ is isomorphic to a proper initial segment of $A$
    \end{enumerate}
\end{theorem}
\begin{proof}
    By the previous lemma, the three cases are mutually exclusive. Thus, we assume that we're neither in the second nor the third case and will prove that we're in the first.\medskip

    We build a strictly monoton map $f:A\hookrightarrow B$ such that the image of $f$ is a proper initial segment of $B$ by transfinite recursion along $A$.

    Note that the empty map is already a strictly monoton map $\emptyset \to B$ whose image $\emptyset$ is an initial segment so that we're done if $A$ is empty and $B$ is non-empty (which it must be if $A=\emptyset$ because we're not in case 2 by assumption).

    \smallskip
    Assume that $f$ is already defined for all $\alpha'<\alpha$ and that $\im(f)=f(A_{<\alpha})$ a proper initial segment of $B$, i.e. $\im(f) = B_{<\beta}$ for some $\beta\in B$.

    Define an extension of $f$ by $\hat{f}(\alpha) := \beta$. That is strictly monoton by construction and
    \[\im(\hat{f}) = \set{\beta} \cup \im(f) = \set{\beta} \cup B_{<\beta} = B_{\leq\beta}\]
    is an initial segment of $B$. It is proper because otherwise $\hat{f}$ would be an isomorphism between $A_{\leq\alpha}$ and all of $B$ so that we'd be in cases 2 (if $\alpha=\max A$) or 3 (otherwise) contrary to our assumption.
\end{proof}