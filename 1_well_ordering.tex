\begin{definition}
    Let $A$ be a well-ordered set. An \emph{initial segment} is a subset $S\subseteq A$ such that
    \[\forall \alpha\in S, \alpha'\in A: \alpha'\leq \alpha \implies \alpha'\in S\]
\end{definition}

\begin{lemma}
    Let $A$ be any well-ordered set.
    \begin{enumerate}
        \item  $\emptyset$ and $A$ itself are initial segments. Sets of the form $A_{<\alpha}$ and $A_{\leq\alpha}$ are initial segments.
        \item All \emph{proper}, initial segments of $A$ are of the form $A_{<\alpha}$ for a unique $\alpha\in A$, namely the smallest element of $A\setminus S$.
        \item $ \alpha'\leq\alpha \iff A_{<\alpha'} \subseteq A_{<\alpha}$. In particular: The set of initial segments
        \[ A^+ = \Set{A_{<\alpha} | \alpha\in A}\cup\set{A} \]
        is itself well-ordered w.r.t.\ to inclusion and $A$ is its greatest element.
    \end{enumerate}
\end{lemma}

\begin{lemma}[Transfinite induction along well-ordered sets]
    Let $A$ be a well-ordered set and $\phi$ a formula with a free variable.
    \begin{enumerate}
        \item If $\phi$ satisfies
        \[(\forall \alpha'<\alpha : \phi(\alpha')) \implies \phi(\alpha)\]
        for all $\alpha\in A$, then $\forall\alpha\in A: \phi(\alpha)$ holds.
        \item If $\phi$ satisfies
        \begin{enumerate}
            \item $\phi(0)$ holds
            \item If $\phi(\alpha')$ holds and $\alpha$ is the successor of $\alpha'$, then $\phi(\alpha)$ holds too.
            \item If $\alpha$ is a limit and $\phi(\alpha')$ holds for all $\alpha'<\alpha$, then $\phi(\alpha)$ holds too.
        \end{enumerate}
        then $\forall\alpha\in A: \phi(\alpha)$ holds.
    \end{enumerate}
\end{lemma}
\begin{proof}
    Obviously, it is sufficient to prove a., since b. is just a restatement.

    Let $F:=\Set{\alpha\in A | \neg\phi(\alpha)}$. Assume $F\neq\emptyset$ and let $\alpha$ be its smallest element. Then $\phi(\alpha')$ holds for all $\alpha'<\alpha$. But then by assumption on $\phi$, $\phi(\alpha)$ also holds contradicting to the minimality of $\alpha$.
\end{proof}

\begin{lemma}
    \begin{enumerate}
        \item For any two well-ordered sets $A,B$ there is at most one isomorphism $A\to B$.
        \item The only initial segment of $A$ isomorphic to $A$ is $A$ itself. In particular: All initial segments of $A$ are pairwise non-isomorphic.
    \end{enumerate}
\end{lemma}
\begin{proof}
    a. Transfinite induction along $A$: Let $f,g:A\to B$ be two isomorphism, $\alpha\in A$ and assume $f(\beta)=g(\beta)$ for all $\beta < \alpha$. Set $A':=A_{<\alpha}$.

    Then by construction $\alpha$ is the smallest element of $A\setminus A'$ and therefore the isomorphisms $f$ and $g$ must map it to the smallest element of $B\setminus f(A')$ and $B\setminus g(A')$ respectively. But by assumption, $f(A')=g(A')$ so that $f(\alpha)=g(\alpha)$.

    Therefore, $f(\alpha)=g(\alpha)$ for all $\alpha\in A$.

    \medskip
    b. Assume $f: A\to A_0$ is an isomorphism onto an initial segment $A_0\subsetneq A$. If there was an $\alpha\in A\setminus A_0$, then $f(\alpha) < \alpha$. But then by induction $f^{n+1}(\alpha)<f^n(\alpha)$ for all $n\in\IN$ so that we get an infinite descending sequence in $A$ which does not exist because $A$ is well-ordered. Therefore $A_0=A$.
\end{proof}

\begin{remark}
    This means that we can strengthen the previous observation: $\alpha'\leq\alpha$ iff $A_{<\alpha'}$ is merely isomorphic to an initial segment of $A_{<\alpha}$.
\end{remark}

\begin{lemma}[Transfinite recursion along well-ordered sets]
Let $A$ be a well-ordered set and $B$ an arbitrary set.
\begin{enumerate}
    \item Let
    \[g: \text{part.Maps}(A\to B)\to B\]
    Then there is a unique function $f: A\to B$ such that
    \[\forall\alpha: f(\alpha) = g(f_{|A_{<\alpha}})\]
    \item Let $h: P(B)\to B$. Then there is a unique function $f: A\to B$ such that
    \[\forall\alpha: f(\alpha) = h(f(A_{<\alpha}))\]
\end{enumerate}
\end{lemma}
\begin{remark}
    In a recursion over $A=\IN$, we might define $f(n)$ by any computation $g$ that may depend on some or all of $f(0), f(1), \ldots, f(n-1)$, i.e.\ any finite-length sequence of values in $B$.

    The appropriate generalisation to sequences with well-ordered index sets would be something like $\bigcup_{S\in A^+\setminus\set{A}}Maps(S\to B)$, i.e.\ certain partial maps $A\to B$.

    Note that the set of partial maps $A\to B$ is may be identified with the set
    \[\Set{G\in P(A\times B) | \forall x\in A\forall y,y'\in B: (x,y)\in G \wedge (x,y')\in G \implies y=y'}\]
\end{remark}
\begin{proof}
    It is sufficient to prove a., because b. is the special case where $g$ is the function $g(f):=h(\im(f))$.

    \medskip
    The idea is to apply transfinite induction to the statement \enquote{$f$ is uniquely defined at $\alpha$}.
    \medskip
    Consider
    \[D:=\Set{S\in A^+ | \exists!S\xrightarrow{f_S} B: \forall\alpha\in S: f(\alpha) = g(f_{|A_{<\alpha}})} \]

    We prove $D=A^+$ by transfinite induction along $A^+$:

    \smallskip
    Step 1. $\emptyset\in D$ because the empty map has that property.

    \smallskip
    Step 2: If $S=A_{<\alpha} \in D$, then $\tilde{S}=A_{\leq\alpha}\in D$ as well, because
    \[f_{\tilde{S}}(\beta) = \begin{cases}
        g(f_S) & \beta = \alpha \\
        f_S(\beta) & \beta<\alpha
    \end{cases}\]
    is the unique extension of $f_S$ to $\tilde{S}$ with the required property.

    \smallskip
    Step 3: $D$ is closed under unions.

    If $S$ is in $D$, then any smaller initial segment $S'\subseteq S$ is also in $D$, because the restriction of $f_S$ also has the property. In particular $f_{S'} = (f_S)_{|S'}$, i.e.\ the functions are all compatible with each another.

    That means that for any $D' \subseteq D$, $S:=\bigcup_{S'\in D'} S'$ is an initial segment and $f_S:=\bigcup_{S\in S'} f_{S'}$ is the unique function with the required property.

    \medskip
    Now note that the successor in $A^+$ of a proper initial segment $A_{<\alpha}$ is $A_{\leq\alpha}$. Therefore, steps 1, 2, and 3 correspond to the 0-case, the successor-case, and the limit-case respectively and transfinite inductions shows that all of $A^+$ is contained in $D$. In particular $A\in D$.
\end{proof}

\begin{theorem}[\enquote{The class of well ordered sets is well-ordered}]
    Given two well-ordered sets $A,B$ exactly one of the following happens:
    \begin{enumerate}
        \item $A$ is isomorphic to a proper initial segment of $B$
        \item $A\isomorphic B$
        \item $B$ is isomorphic to a proper initial segment of $A$
    \end{enumerate}
\end{theorem}
\begin{proof}
    By the previous lemma, the three cases are mutually exclusive. Thus, we assume that we're neither in the second nor the third case and will prove that we're in the first.\medskip

    We build a strictly monoton map $f:A\hookrightarrow B$ such that the image of $f$ is a proper initial segment of $B$ by transfinite recursion along $A$.

    Note that the empty map is already a strictly monoton map $\emptyset \to B$ whose image $\emptyset$ is an initial segment so that we're done if $A$ is empty and $B$ is non-empty (which it must be if $A=\emptyset$ because we're not in case 2 by assumption).

    \smallskip
    Assume that $f$ is already defined for all $\alpha'<\alpha$ and that $\im(f)=f(A_{<\alpha})$ a proper initial segment of $B$, i.e. $\im(f) = B_{<\beta}$ for some $\beta\in B$.

    Define an extension of $f$ by $\hat{f}(\alpha) := \beta$. That is strictly monoton by construction and
    \[\im(\hat{f}) = \set{\beta} \cup \im(f) = \set{\beta} \cup B_{<\beta} = B_{\leq\beta}\]
    is an initial segment of $B$. It is proper because otherwise $\hat{f}$ would be an isomorphism between $A_{\leq\alpha}$ and all of $B$ so that we'd be in cases 2 (if $\alpha=\max A$) or 3 (otherwise) contrary to our assumption.
\end{proof}